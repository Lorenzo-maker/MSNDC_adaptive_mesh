\section*{Conclusion}
A $\pnh$ mesh refinement method for numerical optimal control has been developed. In particular, the $\pnh$ algorithm presents an innovative approach in which the degree of the approximating polynomial can assume distinct values for each state vector component and for each finite element.

The examples highlight the advantages of this approach  especially for the final mesh. In fact, reaching the tolerance $\epsilon$, the final mesh of $\pnh$ reduces the overall number of variables in NLP, hence increase the computational efficiency. These advantages are more pronounced when the states number increases (as in example 2 and 3). 
Furthermore, the solutions difference between this new method and the one proposed in~\cite{Patterson:OCAM:2015} are, in the worst case, 6 order of magnitude smaller than any state or control. This means that a more conservative approach where each state follows the worst state in terms of $p$-refinement, as in the $ph$ method, is unnecessary.  