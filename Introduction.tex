\section*{INTRODUCTION}

Over the past 25 years, direct collocation methods have become popular for the numerical solution of nonlinear optimal control problems, see e.g.~\cite{Fahroo:JGCD:2002,Elnager:TAC:1995,Fahroo:JAS:2000,Gong:AAS:2006}. In direct collocation methods, state and control functions are discretized at a set of approprietely chosen points in the time horizon considered. Then, the continuous time optimal control problem (OCP) is \emph{transcribed} to a finite-dimensional nonlinear program (NLP), and the NLP is coded and solved using different available software packages, such as~\cite{casadi:MPC:2019, Rao:TMS:2010,GPOPSII:TMS:2013} typically interfaced with Interior-Point~\cite{Biegler:CCE:2009} or SQP~\cite{SNOPT:SIAMReview:2005} back-end solvers.

Qui si possono citare approcci molto diversi dal nostro per effettuare il raffinamento della soluzione, tipo [multi-resolution, wavelets, etc.].

Poi partiamo con quelli del nostro filone.

Recently, a great deal of research effort has been spent in the class of Gauss quadrature orthogonal collocation methods []. In these approaches, the state is typically approximated using a Lagrange polynomial where the support points are chosen as Gauss quadrature points. The most developed employ either Gauss-Legendre or Gauss-Radau points. Gauss quadrature collocation methods have originally been implemented as p methods with a single interval for the whole time horizon. Convergence of these methods was achieved by increasing the degree of the approximating polynomial. Their converge rate is exponential for problems with smooth and well-behaved solutions []. On the contrary, h methods, where convergence is sought by increasing the number of mesh intervals, work well for problems whose solutions are non smooth, since even a low-order polynomial approximation can capture wild solution behaviors on an extremely fine mesh. However, their converge rate is slow for problems with smooth solutions as compared to p methods.
With the goal of taking the best of both worlds hp methods have been developed [finite elements for solving PDE] and OCP [Rao, Darby, Hager]. Qui l'ordine è importante perché prima cercano di infittire la mesh e poi aumentare il grado del polinomio. Tuttavia questi metodi [Rao, Darby, Hager] hanno dimostrato bassa efficienza computazionale nella soluzione degli NLP. Inoltre, come affermato dagli stessi autori, il problema di questo approccio hp è che creates a great deal of noise in the error estimate, thereby making these approaches computationally intractable when high-accuracy solutions are desired. Furthermore, the error estimate defined for these methods does not take advantage of the exponential convergence rate of a Gaussian quadrature collocation method.
Da qui la motivazione che ha portato gli stessi autori a sviluppare un metodo diverso, definito ph method, dove la procedure di mesh refinement è ribaltata e che porta una serie di vantaggi.

A questo punto ci inseriamo noi sulle motivazioni che ci hanno spinto a sviluppare un metodo che differenzi il grado del polinomio per ogni stato. Questa cosa ha senso sia perché nell'approccio di Rao, da cui abbiamo preso spunto, loro forse troppo cautelativo portano l'ordine di tutti polinomi al massimo richiesto da quello che sta peggio. Questo ha un vantaggio nel senso che la gestione di cosa succede in ogni elemento è più semplice. Tuttavia si ha uno spreco di risorse, dal momento che per rappresentare l'andamento di una soluzione per una certa componente dello stato, in una data serie di elementi finiti, può essere anche sufficiente un grado più basso di quello necessario per un altra componente dello stato.

