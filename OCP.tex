\section{OPTIMAL CONTROL PROBLEM}

Considering that the $\pnh$ adaptive refinement algorithm is applied to a general optimal control problem (OCP), in this section the structure and the notation of this kind of problem is analyzed.


The generic form of an OCP is as follows
\begin{equation}\label{OCP}
	\begin{split}
		&\underset{X,U}{\text{minimize}}\hspace{8mm} J = \int_{0}^{T}l(X(t),U(t))\,dt + E(X(T))\\
		&\begin{split}
			\text{subject to} \hspace{8mm} & \dot{X} -  f(X(t),U(t))			= 0, \hspace{5mm} t \in[0,T]\\
			& h(X(t),U(t)) \leq 0,  \hspace{12mm} t \in[0,T]\\
			& r(X(T),U(T)) \leq 0,
		\end{split}		
	\end{split}
\end{equation}

where $t$ is the independent variable (time), $T$ is the terminal time, $X \in \Rnx$, $U \in \Rnu$ are the states,and the controls, respectively, and $f (\cdot)$ is the dynamic vector field.

The OCP is characterized by an integral cost function, with a running cost $l$, a terminal cost $E(X(T))$, and three kinds of constraints: dynamics $\dot{X}-f(\cdot) \in \Rng$, inequality $h(\cdot) \in \Rnh$, and terminal constraints $r(\cdot) \in \Rnr$.
The solution of the OCP  consists of the controls $U^{*}(t)$ and the states $X^*(t)$  minimizing the integral cost function while satisfying the above constraints.


