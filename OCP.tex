\section*{OPTIMAL CONTROL PROBLEM}

In this work, the $\pnh$ adaptive refinement algorithm is applied to a general optimal control problem (OCP). Therefore, with the aim to introduced the notation used in the rest of the paper, in this section the overall structure of an OCP and its main components are briefly recalled.


The general form of an OCP can be written as
%\begin{equation}\label{OCP}
%	\begin{split}
%		&\underset{X,U}{\text{minimize}}\hspace{8mm} J = \int_{0}^{T}l(X(t),U(t))\,dt + E(X(T))\\
%		&\begin{split}
%			\text{subject to} \hspace{8mm} & \dot{X} -  f(X(t),U(t))			= 0, \hspace{5mm} t \in[0,T]\\
%			& h(X(t),U(t)) \leq 0,  \hspace{11.5mm} t \in[0,T]\\
%			& r(X(T),U(T)) \leq 0,
%		\end{split}		
%	\end{split}
%\end{equation}
\begin{subequations}\label{OCP}
	\begin{align}
	\underset{X,U}{\text{minimize}}\hspace{8mm} &J = \int_{0}^{T}g(X(t),U(t))\,dt + E(X(T)) \label{eq:cost}\\
	\text{subject to} \hspace{8mm} &\dot{X}(t) -  f(X(t),U(t)) = 0, \hspace{5mm} t \in[0,T] \label{eq:dyn}\\
	& h(X(t),U(t)) \leq 0,  \hspace{11.5mm} t \in[0,T] \label{eq:path}\\
	& r(X(T),U(T)) \leq 0, \label{eq:terminal}		
	\end{align}
\end{subequations}

where $t$ is the independent variable (time), $T$ is the terminal time, $X \in \Rnx$ and $U \in \Rnu$ are the states and the controls, respectively, and $f (\cdot)$ is the dynamic vector field.

The OCP is characterized by an integral cost function, with a running cost $g$, a terminal cost $E(X(T))$, and three kinds of constraints: dynamics $\dot{X}-f(\cdot) \in \Rnx$, inequality $h(\cdot) \in \Rnh$, and terminal constraints $r(\cdot) \in \Rnr$.
The solution of the OCP  consists of the controls $U^{*}(t)$ and the states $X^*(t)$  minimizing the integral cost function while satisfying the above constraints.


