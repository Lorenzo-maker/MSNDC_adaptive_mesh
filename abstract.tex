\begin{abstract}
	{\it This paper presents an automatic procedure to enhance the accuracy of the numerical solution of an optimal control problem (OCP) discretized via direct collocation at Gauss-Legendre points. First, a numerical solution is obtained by solving a nonlinear program (NLP). Then, the method evaluates its accuracy and adaptively changes both the degree of the approximating polynomial within each mesh interval and the number of mesh intervals until a prescribed accuracy is met. %, thus following a ph-strategy.
The number of mesh intervals is increased for all state vector components alike, in a classical fashion. Instead, improving on state-of-the-art procedures, the degrees of the polynomials approximating the different components of the state vector are allowed to assume, in each finite element, distinct values. This explains the $\pnh$ definition, where $n$ is the state dimension. Instead, in the literature, the degree is always raised to the highest order for all the state components, with a clear waste of resources.
		%First, component-wise state error estimates are computed based on the difference between the Lagrange polynomial approximation and a corresponding estimate obtained by quadrature of the dynamics at Gauss-Legendre points within each finite element. Then, the derived error estimates are employed to establish whether the degree of the approximating polynomial should be increased or the mesh interval should be split into subintervals. The first option is selected if the degree suggested by the method remains below a given threshold for all state components. Otherwise, a finer mesh is created in those intervals where excessive error is registered. The NLP problem is solved again with the enhanced mesh and the evaluation procedure described is repeated. The process stops when a solution meets the specified error tolerance from the outset.
		
		Numerical tests on three OCP problems highlight that, under the same maximum allowable error, by independently selecting the degree of the polynomial for each state, our method effectively picks lower degrees for some of the states, thus reducing the overall number of variables in the NLP. Accordingly, various advantages are brought about, the most remarkable being: (i) an increased computational efficiency for the final enhanced mesh with solution accuracy still within the specified tolerance, (ii) a reduced risk of being trapped by local minima due to the reduced NLP size.}
\end{abstract} 